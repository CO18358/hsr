\documentclass[ignorenonframetext,]{beamer}
\setbeamertemplate{caption}[numbered]
\setbeamertemplate{caption label separator}{: }
\setbeamercolor{caption name}{fg=normal text.fg}
\beamertemplatenavigationsymbolsempty
\usepackage{lmodern}
\usepackage[british]{babel}
\usepackage{amssymb,amsmath}
\usepackage{algorithm}
\usepackage{algpseudocode}
\usepackage{pgfplots}
\usepackage{ifxetex,ifluatex}
\usepackage{fixltx2e} % provides \textsubscript
\ifnum 0\ifxetex 1\fi\ifluatex 1\fi=0 % if pdftex
  \usepackage[T1]{fontenc}
  \usepackage[utf8]{inputenc}
\else % if luatex or xelatex
  \ifxetex
    \usepackage{mathspec}
  \else
    \usepackage{fontspec}
  \fi
  \defaultfontfeatures{Ligatures=TeX,Scale=MatchLowercase}
\fi
% use upquote if available, for straight quotes in verbatim environments
\IfFileExists{upquote.sty}{\usepackage{upquote}}{}
% use microtype if available
\IfFileExists{microtype.sty}{%
\usepackage{microtype}
\UseMicrotypeSet[protrusion]{basicmath} % disable protrusion for tt fonts
}{}
\newif\ifbibliography

% Prevent slide breaks in the middle of a paragraph:
\widowpenalties 1 10000
\raggedbottom

% \AtBeginPart{
%   \let\insertpartnumber\relax
%   \let\partname\relax
%   \frame{\partpage}
% }
% \AtBeginSection{
%   \ifbibliography
%   \else
%     \let\insertsectionnumber\relax
%     \let\sectionname\relax
%     \frame{\sectionpage}
%   \fi
% }
% \AtBeginSubsection{
%   \let\insertsubsectionnumber\relax
%   \let\subsectionname\relax
%   \frame{\subsectionpage}
% }

\setlength{\emergencystretch}{3em}  % prevent overfull lines
\providecommand{\tightlist}{%
  \setlength{\itemsep}{0pt}\setlength{\parskip}{0pt}}
\setcounter{secnumdepth}{0}

\usepackage{sympytex}
\usepackage{listings}

\title{Hidden Surface Removal}
\author{André Grüning, University of Surrey}
\date{\today}

\begin{document}
\frame{\titlepage}

\begin{frame}{Hidden Surface Removal}

{[}from Kleinberg, Tardos: ``Algorithm Design'' (our textbook){]}

\begin{block}{Background}

\emph{Hidden Surface Removal} is a problem in Computer graphics that
scarcely needs an introduction: when Woody is standing in front of Buzz,
you should be able to see Woody but not Buzz; when Buzz is standing in
front of Woody,\dots 

\end{block}

\end{frame}

\begin{frame}

\begin{block}{Elaboration}

The magic of hidden surface removal is that you can often compute things
faster than your intuition suggests. Here's a clean geometric example to
illustrate a basic speed-up that can be achieved. You are given \(n\)
nonvertical lines in the plane, labelled \(l_1, \dots l_n\) with the
\(i\) line specified by \(l_i(x) := m_ix + b_i\). We will make the
(simplifying) assumption that no three of the lines meet at a single
point. We will also assume that no two lines have the same slope, ie $m_i \neq m_j, \forall i\neq j$.

Which of the lines are visible by an observer high up in the \(y\)
direction (and hence need to be rendered)?

\begin{itemize}[<+->]
\tightlist
\item
  We say that line \(l_i\) is \emph{uppermost} at a given
  \(x\)-coordinate \(x_0\) if its \(y\)-coordinate is greater than
  \(y\)-coordinates of all other lines at \(x_0\).
\item
  We say line \(l_i\) is \emph{visible} if there is some
  \(x\)-coordinate at which it is uppermost -- intuitively some portion
  of it can be seen if you look down from \(y \to \infty\).
\end{itemize}

\end{block}

\end{frame}

\begin{frame}{Our programme}

\begin{block}{Our programme}

\begin{itemize}[<+->]
\tightlist
\item
  Understand the problem.
\item
  Try to find a brute force algorithm to gain a deeper understanding of the problem.
 \item
  Implementation: What data structure do we need to present a line?
\item
  Give an algorithm that takes \(n\) lines and in \(O(n\log n)\) time
  returns all of the ones that are visible.
\end{itemize}

\end{block}

\end{frame}

\section{Understand the problem}

\begin{frame}{Understand the problem}

\begin{itemize}
    \item We draw a couple of lines to understand what it means to be uppermost and to be visible. 
    \item (It's always good to try to find some visualisation to understand a new problem - and you should make that a habit.)
\end{itemize}
    
Before proceeding from this point we all should have a good understanding of what being uppermost and being visible means and why this is related to Hidden Surface Removal. If not, draw a few more different arrangements of straight lines which are partially or fully obscured by others when viewed from the top of your figures
\end{frame}

\section{Brute Force}

\begin{frame}{Brute Force Approach}

  Just take the problem literally:
  \begin{enumerate}
  \item To be visible, a line must be uppermost at least one point.
  \item Given a point $x_0$ a line is uppermost if it has the greatest $y$-coordinate.
  \item So we need to check at all points $x$ in a given range which line is uppermost.
  \end{enumerate}
\end{frame}

\begin{frame}{Brute Force Pseudocode}

  \begin{algorithm}[H]
    \caption{Removing Hidden Surfaces Brute Force}
    \begin{algorithmic}
      \Procedure{CalcVisibleSurfaces}{$L$: list of lines, $R \subset \mathbf{R} $: interval of interest for rendering}
      \State{$V = \emptyset$} \Comment{Set of visible lines}
      \For{$x \in R$}
      \State{$\text{uppermost} = \max\arg_{l \in L} l(y)$  \Comment{ $O(n)$ time}}
      \State{Add uppermost to $V$}
      \EndFor
      \Return{$V$}
      \EndProcedure
    \end{algorithmic}
  \end{algorithm}
  \begin{itemize}
  \item Running time: Number of points in $R$ times O(n).
  \item But what is the size of $R$? If $R$ is an interval then there are an infinite number of points. $\Rightarrow$ Infinite running time\footnote{Of course if we only have finite resolution screen to render for, $R$ is not an interval in $\mathbf(R)$, but a set of discrete points -- eg the $x$-coordinates of 4000 pixels (assuming the screen is 4000 pixels wide. But still: what if we want to calculate this for an arbitrary yet unknown resolution, or just -- as we are in this module -- want to know the answers in abstractu.}
  \end{itemize}
\end{frame}

\begin{frame}{Frame Title}
    \begin{itemize}
        \item Clearly, this brute force algorithms is not feasible.
        \item Do we really need to calculate which line is uppermost at all $x \in R$?
        \item Or is there a smaller set of somehow "special" points?
        \begin{itemize}
            \item Q: Can what is the uppermost line change at any point, or only at special points? And what happens between these points?
        \end{itemize}
    \end{itemize}
\end{frame}

\section{Refined Brute Force}
\begin{frame}{Refined Brute Force}

\begin{itemize}[<+->]
    \item Observation: Uppermost line can only change at intersection points of any two lines.
    \item Therefore at any point \emph{between} two intersections the uppermost line will not change.
    \item Idea: Check only which is the uppermost line between any two intersections.\footnote{We could also check at the intersections points themselves, but that would need more case distinctions (why?)}
    \item What would be the running time of this refined brute-force algorithm?
        \begin{itemize}
            \item For $n$ lines there are $O(n^2)$ intersections\footnote{To be more precise it is $ (n-1) + (n-2) + (n-3) + \cdots + 2 + 1 = \sum_{i = 1}^{i < n} i = \frac{(n-1)(n-2)}{2}$ intersections.}, and therefore also $O(n^2)$ intervals between two neighbouring intersections.
            \item For each of these $O(n^2)$ intervals e need to pick a point to calculate the $y$ coordinate of $n$ lines.
            \item[$\Rightarrow$] $n \times O(n^2) = O(n^3)$ steps.
        \end{itemize}
        \end{itemize}
\end{frame}

\begin{frame}{What is missing?}
    What does this outline of the algorithm not specify?
    \begin{itemize}
        \item How to find the intervals.
        \item That the intersection points need to be sorted.
        \item Which point to choose in an interval.
        \item \dots
    \end{itemize}
\end{frame}

\begin{frame}{Refined Brute Force Pseudocode}

\begin{algorithm}[H]
\caption{Refined Brute Force}
\begin{algorithmic}
    \Procedure{CalcVisibleSurfaces}{L: list of lines}
    \State{$I$: list of $x$-coordinates of intersections}
        \For{any pair of indices $i,j$ with $i < j$} \State{\Comment{ $O(n^2)$}}
        \State{Calculate the $x$-coordinate of intersection between $l_i$ and $l_j$}
        \State{Add this $x$ to $I$}
        \EndFor
        \State{$V = []$: list of visible lines}
        \State{Sort $I$ by increasing $x$} \Comment {$O(n^2\log n^2)$}
        \For{pairs of adjacent $x_1,x_2$ in $I$} \Comment{$O(n^2)$}
        \State{$x_0 = \frac{x_1 + x_2}{2}$}
        \State{$uppermost = \max\arg_{l \in L} l(x_0)$} \Comment{$O(n)$}
        \State{Add $uppermost$ to $V$}
        \EndFor
        \State{Remove duplicates from $V$} \Comment{$O(n \log n)$}
        \Return{V}
    \EndProcedure
\end{algorithmic}
\end{algorithm}
Note: The code does not depend on the size/length of the interval $R$ any longer.
\end{frame}

\begin{frame}{Problems with this code}

\begin{itemize}%[<+->]
    \item Is this code complete? 
      \begin{itemize}
      \item Actually a corner case is missing from the code. Which?
      \end{itemize}
    \item Test it on a small example!
    \item How do we remove duplicates?
    \item Why do we take a list and remove duplicates, and not a set?
\end{itemize}
\end{frame}

\begin{frame}{Implementation of the refined brute force algorithm.}
    We firstly need to think about the data structure to represent a
    line $l(x) = mx+b$ in code. Fields for a line object are from the
    above: 
    \begin{itemize}
    \item A line \emph{has} a slope $m$.
    \item A line \emph{has} an offset $b$
    \end{itemize}
    and we need a line object to do the following functions:
    \begin{itemize}
    \item return its value $y = l(x)$ for a given point $x$: method with signature \texttt{y(x)}
    \item return the $x$-coordinate of the intersection with another line: method with signature \texttt{intersection(Line other)}
    \end{itemize}
    The implementation of \texttt{y(x)} is straight forward. For
    method \texttt{intersection} we need to know how to calculate
    $x$-coordinate $x_0$ of the 
    intersection of two lines $l_1(x) = m_1x + b_1$ and $l_2(x) = m_2 +
    b_2$. Basic algebra yields
    \begin{equation}
      x_0 = \frac{m_1 - m_2}{b_2 - b_1}
    \end{equation}
\end{frame}

\begin{frame}{Data structure representing a line}

\tiny
\lstinputlisting[language=Python, breaklines=true, firstline=4,
lastline=33 ]{HiddenSurfaceRemoval.py}
\end{frame}

\begin{frame}{Refined Brute Force -- Python implementation}
\tiny
\lstinputlisting[language=Python, breaklines=true, firstline=35,
lastline=62 ]{HiddenSurfaceRemoval.py}
\end{frame}

\begin{frame}{Running the algorithm}

We have now at least code that we can run on a set of straight lines
in order to get a better idea of what geometric properties might mark
the visible lines.

\begin{block}{Code to generate 100 random lines and run the algorithm}
\tiny
\lstinputlisting[language=Python, breaklines=true, firstline=157,
lastline=167 ]{HiddenSurfaceRemoval.py}
\end{block}
\end{frame}



\begin{frame}[fragile]{100 straight lines}

  \begin{sympysilent}
    import HiddenSurfaceRemoval as hsr
    pL, pV, pR, visible = hsr.hundred_lines()
  \end{sympysilent}



  This is what 100 random straight lines look like:
  \begin{figure}
    \caption{100 random straight lines. Which are visible?}
    \sympyplot[width=\textwidth]{pL}
    \centering
  \end{figure}
\end{frame}

\begin{frame}{Visible Lines}

  \begin{figure}
    \caption{Visible lines among the random lines -- typically in the
      order of 10. (Note: not all sections of these lines are visible.}
    \sympyplot[width=\textwidth]{pV  }
    \centering
  \end{figure}

\end{frame}

\begin{frame}[fragile]{Visible Lines}

  \begin{sympysilent}
    pL.extend(pV)
  \end{sympysilent}

  \begin{figure}
    \caption{All lines with the visible ones in red.   On visual inspection, it appears the the lines with the steepest
  positive or negative slope tend to be visible. Perhaps that a cue
  for a later fast algorithm?}
    \sympyplot[width=\textwidth]{pL}
    \centering
  \end{figure}
\end{frame}

\begin{frame}{Towards a fast algorithm}

\begin{itemize}[<+->]
    \item We got from the brute force algorithm to the refined
      algorithm by using geometric properties of the problem. 
    \item Can we find other properties of the problem that help reduce
      time complexity? 
    \item Do we really need to calculate the $y$-coordinates of
      \emph{all} lines at \emph{all} intersection points? 
    \item For both a Dynamic Programming approach or a Divide and
      Conquer approach\footnote{At this stage we don't know yet which
        of them could lead us to an algorithm for this problem.}, 
      what would be any base cases?
    \item Does it make sense to order the lines in a certain order?
      Which orders would be natural? 
\end{itemize}
\end{frame}

\begin{frame}{The simplest cases -- 0 and 1 lines}

In search of what could be base cases for a recursive algorithms
(either D\&C or Dynamic Programming), it makes sense to look at the
simplest cases:

\begin{itemize}
\item No lines: then none are visible
\item 1 line: then it's also visible
\end{itemize}
So these cases are fairly trivial.
\end{frame}

\begin{frame}{The simplest cases -- 2 lines}

Two lines: As we assume that no two lines are parallel, they must
intersect somewhere, and are both visible. So that's trivial as well.

\begin{tikzpicture}%[>=latex]
    \begin{axis}[
    %with=6cm,
    axis lines=middle,
    grid,
    grid style={densely dashed}, 
    xlabel=\(x\), ylabel=\(y\),
    every axis plot/.style={very thick},
    ]
    \addplot coordinates { (8,2) (-2,-3)};
    \addplot coordinates { (-2,-2) (8,1)};
    %\addplot coordinates { (-2,2) (8,-1)};
    \end{axis}
\end{tikzpicture}

But we can perhaps learn something useful for later: To the left of
the intersection point, the line with the smaller slope is uppermost
and to the right of it the one with the greater slope.
\end{frame}

\begin{frame}[fragile]{Three lines -- which are visible?}

  \begin{columns}
    \begin{column}{0.5\textwidth}
      \begin{tikzpicture}[scale=0.8]%[>=latex]
        \begin{axis}[
          % with=6cm,
          axis lines=middle,
          grid,
          grid style={densely dashed}, 
          xlabel=\(x\), ylabel=\(y\),
          every axis plot/.style={very thick},
          ]
          \addplot coordinates { (8,2) (-2,-3)};
          \addplot coordinates { (-2,-1.5) (8,1.5)};
          \addplot coordinates { (-2,2) (8,-1)};
        \end{axis}
      \end{tikzpicture}
      Three lines visible
  \end{column}
  \pause
  \begin{column}{0.5\textwidth}
    
    \centering
    
    \begin{tikzpicture}[scale=0.8]%[>=latex]
      \begin{axis}[
        % with=6cm,
        axis lines=middle,
        grid,
        grid style={densely dashed}, 
        xlabel=\(x\), ylabel=\(y\),
        every axis plot/.style={very thick},
        ]
        \addplot coordinates { (8,2) (-2,-3)};
        \addplot coordinates { (-2,-2) (8,1)};
        \addplot coordinates { (-2,2) (8,-1)};
      \end{axis}
    \end{tikzpicture}
    Two lines visible
  \end{column}
\end{columns}
We observe (by drawing many more arrangements of three lines):
\scriptsize
\begin{itemize}[<-+>]
\item The lines with the greatest slope and the one with the smallest
  slopes are always visible. Let's call them $L_0$ and $L_2$.
\item The line $L_1$ with the middle slope is only visible if at the
  intersection $I_{02}$ of $L_0$ and $L_2$ it is above this
  intersection.\footnote{\tiny If $L_1$ is below $I_{02}$, then to the right of
    $I_{02}$, $L_2$ is above it and due to its greater slope will
    always be above. To the left of $I_{02}$, $L_0$ is above, and due
    to its smaller (more negative slope) will always stay above it.}
\end{itemize}
\end{frame}


\begin{frame}{What have we got so far?}

  \begin{block}{What we have got so far.}
    \begin{itemize}
    \item The slope of the lines is important.
    \item Lines with greatest and smallest slope are always visible
      (namely far to the right or left).
    \item The position of a line wrt to the intersection of others is
      important (see case of 3 lines).
    \item We can deal with 0,1,2,3 lines already.
    \end{itemize}
  \end{block}
  \pause
  \begin{block}{Approach}
    From the above we
    \begin{itemize}
    \item Sort lines by increasing slope.
    \item Try a D\&C approach with 0,1,2,3 lines as base case
    \end{itemize}
  \end{block}
\end{frame}


\begin{frame}{Follow the D\&C template}

  \begin{algorithm}[H]
    \caption{Nascent Algorithm D\&C}
    \begin{algorithmic}
      \footnotesize

     \State{Sort list of lines $L$ by slope $m$, such that $l_0,l_1,\dots l_{n-1} \in
        L$ with $m_0 < m_1 < m_2 <  \cdots < m_{n-1}$.}
      \Procedure{D\&C}{$L$: list of $n$ lines}
      \If{$n < 4$}
     \State{Deal with the base cases $n=0,1,2,3$ to get set of visible lines $V$.}
     \State{\Return{set of visible lines: $V$}}
      \EndIf
      \State{split $L$ into \texttt{Left} and \texttt{Right} halves}
      \State{visible $V_{left} =$ \Call{D\&C}{\texttt{Left}}}
      \State{visible $V_{right} =$ \Call{D\&C}{\texttt{Right}}}
      \State{Somehow merge the separate sets {$V_{left}, V_{right}$}
        into one set $V$.}
     \State{\Return{set of visible lines $V$}}
      \EndProcedure
    \end{algorithmic}
    \end{algorithm}
    \pause
    \footnotesize
    \begin{block}{Big question}
      How do we merge $V_{left}, V_{right}$, and determine whether any
      lines in $V_{left}$ occlude any lines in $V_{right}$ and vice-versa?
    \end{block}
  \end{frame}

\begin{frame}{Nascent Merging of set of visible lines}

  Let us call the nascent function \textsc{MergeVisible}.

  \begin{block}{What we know about \textsc{MergeVisible}}
    \begin{itemize}
    \item It will take two (sorted!) lists \texttt{Left} and
      \texttt{Right} of visible lines as input. 
    \item It must return a (sorted!) list $V$ of the jointly visible
      lines from \texttt{Left} and \texttt{Right}.
    \item All lines in \texttt{Right} are steeper than any line in
      \texttt{Left} because they were sorted before the split.
    \item The first line (with the smallest slope) in \texttt{Left} will
      be visible, as will the last line (with the greatest slope in \texttt{Right}.
    \end{itemize}
  \end{block}

  We should probably make some observations on how a set of visible lines
  (eg from \texttt{Left}) does or does not get occluded by a line that
  has a greater slope (eg one from \texttt{Right}). Can lines from
  \texttt{Right} also be occluded by lines from \texttt{Left}?
\end{frame}


\begin{frame}[fragile]{Properties of sets of visible lines}

  Perhaps we should think about exactly which piece of a line is
  visible, and therefore which piece should be rendered (we can compute these based on our set of visibiles from the refined brute force algorithm)

  \begin{sympysilent}
    pV.extend(pR)
  \end{sympysilent}

  \begin{columns}
    \begin{column}{0.5\textwidth}
      \begin{figure}[H]
        \centering
        \caption{Uppermost sections of the line (ie those section of a line that need to be rendered)}
        \sympyplot[width=\textwidth]{pR}
      \end{figure}
    \end{column}
    \begin{column}{0.5\textwidth}
      \begin{figure}[H]
        \centering

      \caption{Uppermost sections of the line (ie those section of a line that need to be rendered)}
      \sympyplot[width=\textwidth]{pV}
    \end{figure}
  \end{column}
  \end{columns}

\end{frame}

\begin{frame}{Observations of the visible sectios of lines}
  
  We make the following observation wrt to the intersection points of
  the visibile lines:

  \begin{block}{Observation}
    Let $V = [l_1, l_2, \cdots, l_N]$ be the list of visible lines, ordered by increasing slope. 
    Then the intersetion points $[x_i:  x_1 \text{is intersection of
      $l_i$ and $l_{i+1}, i < N$}]$ have increasing $x$-coordinates.
  \end{block}

  \begin{block}{Proof}
    If two lines are both visibile, the region where the one with the
    smaller slope is visible must be to the left of the one with the greater
    slope. Therefore $l_i$ is visible to the left of $x_i$ and
    $l_{i+1}$ to the right of $x_i$. Similarly, $l_{i+1}$ is visible
    to the left of $x_{i+1}$ and $l_{i+1}$ to the right of
    $x_{i+1}$. Therefore $x_{i+1}$ must be to the right of $x_i$, ie
    $x_i < x_{i+1}$.
  \end{block}
\end{frame}
\begin{frame}
  \frametitle{More Observations}
  \begin{block}{Corollary}
    For the lines in \texttt{Left}, $l_1$ is uppermost to the left of $x_1$, $l_N$ to the right of
    $x_{N-1}$. Any other line $l_i, 1 < i < N-1$ is visible between
    $x_{i-1}$ and $x_1$. (and analogously for the lines in \texttt{Right})
  \end{block}

  \begin{block}{Corollary}
    The piecewise linear function given by the visible sections of
    lines is convex up, as the next (ie to the right) visibile line
    will have a greater slope. 
  \end{block}
\end{frame}


\begin{frame}{How can two convex-up curves intersect?}

  \begin{itemize}
  \item 
    Back to our D\&C algorithm and the question: Given two lists
    \texttt{LEFT} and \texttt{RIGHT} of
    visible lines (each ordered by slope) and such that the slope of the
    \texttt{RIGHT} lines is always greater than those of \texttt{LEFT},
    which ones are visible in the joint set of lines?
  \item 
    Perhaps it makes sense to look at how two convex-up shapes
    \texttt{LEFT} and \texttt{RIGHT} can intersect where \texttt{RIGHT}
    finishes with a slope that is greater than the one \texttt{LEFT} ended
    with.
  \item 
    For visualising and experimenting with these shapes (do some
    drawings of these convex shapes and their intersection). Ignore that
    the lines are piecewise linear (ie have edges) for a while, and
    look just at two smooth convex-up functions eg $\exp(-x)$ and $\exp(x)$ to which the
    condition of the slopes applies, and shift these two functions
    about.
  \item 
    In order to find the sections in which these curves are visible, we
    need to find their intersection point $x_0$: The left convex-shape
    is visible to left of this intersection point, and the right
    convex-shape is visible to the right of this intersection.
  \end{itemize}
\end{frame}

\begin{frame}{Conclusions about the intersection of two convex-up curves}
  
  \begin{enumerate}
  \item This intersection point is unique.
  \item \texttt{LEFT} is visible to the left of the intersection point.
  \item \texttt{RIGHT} is visible to the right of the intersection point.
  \end{enumerate}
\end{frame}

\begin{frame}{Back to our piecewiese rendered curve}
  \begin{itemize}
  \item The argumentation for the smooth convex curves applies to any
    concex curves, that is also for our piecewise linear curves
    \texttt{LEFT} and \texttt{RIGHT} there is a unique intersection
    point $x_0$, and left to it only sections of lines from \texttt{LEFT}
    are visible, and right from it only sections of lines from
    \texttt{RIGHT} are visible.
  \item Therefore: need to find intersection point $x_0$ of
    \texttt{LEFT} and \texttt{RIGHT}: all visibile lines in
    \texttt{LEFT} that are uppermost to the left of $x_0$ are
    visible, as are all lines in \texttt{RIGHT} that are uppermost to
    the right of $x_0$
  \item Therefore: discard all lines in \texttt{LEFT} right of $x_0$ that were
    visible within \texttt{LEFT}, because they will be occluded by lines from \texttt{RIGHT}.
  \item Therefore: discard all lines in \texttt{RIGHT} to the left of
    $x_0$ that were visible in \texttt{RIGHT}, because they will be
    occluded by lines from \texttt{LEFT}.
  \end{itemize}
  
  \begin{block}{Task}
    How to efficiently find $x_0$?
  \end{block}
\end{frame}

\begin{frame}
  Situaion plotted here with 3 lines from each \texttt{LEFT} and
  \texttt{RIGHT}.
  \begin{figure}
    \centering

    \begin{tikzpicture}[scale=0.8]%[>=latex]
    \begin{axis}[
      % with=6cm,
      axis lines=middle,
      grid,
      grid style={densely dashed}, 
      xlabel=\(x\), ylabel=\(y\),
      every axis plot/.style={very thick},
      ]
      \addplot[red] coordinates { (-1,2) (0,-3)};
      \addplot[red] coordinates { (-1,1) (1,-3)};
      \addplot[red] coordinates { (-1,0) (2,-3)};
      \addplot[blue] coordinates { (2,2) (0,-3)};
      \addplot[blue] coordinates { (2,1) (-1,-3)};
      \addplot[blue] coordinates { (2,-1) (-2,-3)};
    \end{axis}
  \end{tikzpicture}

\caption{In this case the third red line is intersection by the 2 blue line,
  meaining that all lines from LEFT are still visibile, but only 2nd
  and 3rd lines from \texttt{RIGHT}. Now imaging shifting the set of
  blue lines about relative to the red lines, such that a different
  number of them are visible: Eg shifting all blue lines right will
  eventually make them all visible. Shifting them up, down, right will
  lead to different result -- play through all the different
  configuration that may occur.}
\end{figure}
\end{frame}

\begin{frame}{Procedure to find intersection $x_0$}

  \begin{block}{Uppermost lines at any $x$}
    \begin{itemize}
    \item We observed earlier, that the uppermost line within \texttt{LEFT}
      only changes at one of the intersection points $x_i$ of adjacent uppermost
      lines, and then it changes from $l_i$ to $l_{i+1}$. 
      So for each $x$ we can readly find which line is uppermost by
      iterating through the $x_i$ in order until we have the first $x_i
      > x$, then $l_i$ is the uppermost line at $x$. (This takes $O(n)$ time steps.)
    \item Same procedure can be applied to \texttt{RIGHT}: for each
      interval between its intersection points, we can determine which
      line is uppermost by simply iterating through their intersection
      points.
    \item So we can find out for any given $x$ which line is uppermost
      within $O(n)$ times steps from \texttt{LEFT} visible set and which is
      uppermost from the \texttt{RIGHT} visible set (by iterating the
      relevant intersection points.
    \item Let's denote the intersection points of the visible lines from
      th left set as $x_0,\cdots x_{K-2}$ if there are $K$ lines in the
      \texttt{LEFT} set, and the intersection points of visible lines
      from the RIGHT set with $J$ visible lines as $\xi_0, \cdots
      \xi_{J-2}$.
    \end{itemize}
  \end{block}
\end{frame}  

\begin{frame}
  \frametitle{First line from \texttt{LEFT} that is occluded by a line
  from \texttt{RIGHT}}

\begin{itemize}
\item From our earlier reasoning about intersection points of up-convex
  points, there is only one unique intersection $x_0$ between the visible section
  of \texttt{LEFT} and \texttt{RIGHT} lines, and no lines of
  \texttt{LEFT} are uppermost to the right of $x_0$:
  \begin{itemize}
  \item At the intersection $x_0$, two lines $l_i$ and $\lambda_j$
    intersect (with as of yet unknown incides $i$ and $j$), but we can
    say that no line $l_{i^*}$ with $i^* > i$ can be uppermost to the
    left of $x_0$: By construction, $l_i$ is uppermost at $x_0$ (or
    slightly left thereof), ie
    higher than $l_{i^*}$, but also $\lambda_j$ is higher than $x_{i^*}$,
    and due to the smaller slope of $x_{i^*}$ wrt to $\lambda_j$ it
    will never be able to catch up with it to the right of $x_0$, ie
    remain forever occuluded by $\lambda_j$.
  \item However all lines $l_{i^-}$ with $i^- < i$ will remain visible
    to the left of $x_0$ as they were before: The intersection point
    $x_0$ is unique, and it lies to the left of the uppermost section
    of any $l_{i^-}$. 
  \end{itemize}
  Therefore once we know $x_0$, and the index $i$ for the $l_i$ we
  will be able to say exactly which lines from \texttt{LEFT} are
  visible, namely $l_i$ and all $l_{i^-}$ with $i^- < i$. 
\item ( A symmteric arigument for \texttt{RIGHT} shows that all lines
  $\lambda_{j^+}$ with $j^+ \geq j$ will remain visible in the joint
  set of \texttt{LEFT} and \texttt{RIGHT}.)
\end{itemize}

\end{frame}


\begin{frame}
  \frametitle{Find $x_0$? Or rather $i,j$?}
  \begin{itemize}
  \item How can we find out the location of $x_0$ therefore? 
  \item We probably don't need to know its exact location actually. 
  \item It is enough if we now the index $i$ into \texttt{LEFT} (and likewise $j$
    for \texttt{RIGHT}.
  \item According to our argument that if $l_i$ is the line involved
    in the intersection at $x_0$ with a line from \texttt{RIGHT}, the
    intersection $x_i$ of $l_i$ and $l_{i+1}$ must be below the
    uppermost line from \texttt{RIGHT} (otherwise a section of
    $l_{i+1}$ would be visible.), but no line from \texttt{RIGHT} can
    be higher than intersection of $x_{i-1}$  of $l_{i-1}$ and $l_i$
    (otherwise $l_i$ would not be visible).
  \item In other words: $x_i$ is the first intersection within
    \texttt{LEFT} such that a line from \texttt{RIGHT} is uppermost at
    this interseciton.
  \item Therefore we will go through the $x_i$ in order, and for each
    of the $x_i$ check where for the first time a line from \texttt{RIGHT} is higher at
    $x_i$. If so, we have found our index $i$.
  \item (Similarly we can iterate backwards through the $\xi_j$ and
    find the $j$ where for the first time $\xi_j$ is lower than the
    uppermost line in \texttt{LEFT}.
  \end{itemize}
\end{frame}

\begin{frame}
  \frametitle{Still Finding $i$}
  In other words, it is important to find out for all intersection
  points $x_i$ from \texttt{LEFT}, which line from \texttt{RIGHT} is
  uppermost at this point.
  \begin{itemize}
  \item Finding out for a given $x$ which line from \texttt{RIGHT} is
    uppermost takes $O(n)$ as argued before -- but we need to do this
    for $O(n)$ intersection points from \texttt{LEFT}. 
  \item This will take $O(n^2)$ time steps altogether, and by the
    master theorem lead to a running time of $O(n^2)$ (use $a=2$,
    $b=2$, and $c = 2$
  \item This is better than the refined brute force approach with
    $O(n^3)$.
  \item But can we stil do better?
  \end{itemize}
  \begin{block}{Exercise}
    Develop the pseudo-code for this algorithm. You will notice that
    you will have to change the signature of \textsc{MergeVisible} so
    that it not only returns the list of visible lines, but also the
    list of their intersection in order, which is a by product of the
    required calculations in \texttt{MergeVisible} so does not cost
    any extra time in the sense of increasing the big-O time complexity.
  \end{block}
\end{frame}

\begin{frame}{Finding $i$ faster}

  [TBD: The argument on this slides is to woolly and not precise]
  \begin{itemize}
  \item We haven't utilised the fact that both the $x_i$ and $\xi_j$
    are ordered. 
  \item Therefore for a given $x_i$ we only need to consider those
    $\lambda_j$ with $\xi_j < x_i$. (Or similar -- this is only
    correct up to +- 1 -- I need to make this argumet here precise and
    agree with the later source code).
  \item So in fact for $x_i$ we only need to consider those $\xi_j$ that we did
    not already consider for $x_{i-1}$, ie those that are between
    $x_{i-1}$ and $x_{i}$.
  \item So we have two order lists of points $x_i$ and $\xi_j$
    respectively and we need to deal with all $x_{i-1} < \xi_j < x_i$
    before we can proceed to the next $x_{i+1}$.
  \item This is faintly reminisence of the core idea of
    \textsc{MergeSort} and \textsc{InversionCount} where we had to
    deal with to sorted list, but could only deal with the remaining
    smallest item in the one list when we had dealt with all smaller
    items in the other list, where we could use up both lists in just
    $O(n)$ time steps.
  \end{itemize}
\end{frame}

\begin{frame}
  \frametitle{Finding $i$ fast with ideas from \textsc{MergeSort}}
  
  Core idea of going through the intersection points of both lists in
  order. This is again a bit whooly, but the precises indices will
  come from debugging the implemented pseudo-code:

  So this is only a sketch:
  \begin{itemize}
  \item State with $i = 0$ and $j=0$.
  \item Compare $x_i$ and $\xi_j$. 
  \item If $x_i$ is lower than $x_j$, calculate $l_i(x_i)$, and comapre to height
    of $lambda_{j}$ at $x_i$. If $\lambda_{j}$ is higher than $l_i()$
    at this point, return $i$ -- the index of last visible line
    in. Otherwise increase $i$.
    \texttt{LEFT}.
  \item If $x_j$ is lower than $x_i$, increase $j$.
  \item Repeat from the compare step, until all indicies are exhausted.
  \end{itemize}
\end{frame}

\begin{frame}
  \frametitle{Pseudocode for $O(n)$ \textsc{MergeVisible}}
  \begin{algorithmic}
    \Procedure{MergeVisible}{$V_L, V_R$: Lists of visible lines
      separately for left and right halves, in order of increasing slope}
    \EndProcedure
  \end{algorithmic}
\end{frame}

\end{document}


\item Also for each $x_i$ we know which line is uppermost in \texttt{LEFT}
    and what its height is, namely $l_i(x_i) = l_{i+1}(x_i)$, but we
    also know which lines is uppermost at this point in
    \texttt{RIGHT}, say $k_j$. 
  \item Similar to how ww dealt with the 3 lines (base?) case We can
    now compare $y_{L} = l_i(x_i)$ and $y_{R} = k_j(x_i)$. 
  \item If $y_{L} > y_{R}$, then $l_{i+1}$ is visible 
  \item and if $l_{i+1}$ is visible, so will be all $l_s$ with $s <
    i+1$, 
  \item Can we conclude anything about $k_j$? $k_j$ is not visible
    at $x_i$ and to the left of it, but might still become uppermost
    to the right of it.
  \item However $k_{j-1}$ is certainly not visible because at $x_i$ is
    must be under $k_j$ (otherwise it would be the the uppermost in
    \texttt{RIGHT} at $x_i$, and there it is also under $x_i$, and a
    forteriori under all $x_s, s < i$ as $k_{j-1}$ due to the greater
    slope (ie less negative) will decay faster to $-\infty$, than any
    of $l_s, s \leq i$, ie also stay under $l_i$ and its predecessors.
  \item If $y_{L} < y_{R}$ at $x_i$ this means that $k_j$ is visible at least to the
    right of $x_i$, and with it all $k_r, r \geq j$ will also be visible.
  \item Check whether I left some of the ones out, namely if
    intersections from \texttt{LEFT} and \texttt{RIGHT} are not
    alternating.
  \item Clarification $x_0$ is intersection of the separately uppermost lines.
  \end{itemize}
\end{frame}

\end{document}
