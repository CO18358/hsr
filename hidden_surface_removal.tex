\documentclass[ignorenonframetext,]{beamer}
\setbeamertemplate{caption}[numbered]
\setbeamertemplate{caption label separator}{: }
\setbeamercolor{caption name}{fg=normal text.fg}
\beamertemplatenavigationsymbolsempty
\usepackage{lmodern}
\usepackage[british]{babel}
\usepackage{amssymb,amsmath}
\usepackage{algorithm}
\usepackage{algpseudocode}
\usepackage{pgfplots}
\usepackage{ifxetex,ifluatex}
\usepackage{fixltx2e} % provides \textsubscript
\ifnum 0\ifxetex 1\fi\ifluatex 1\fi=0 % if pdftex
  \usepackage[T1]{fontenc}
  \usepackage[utf8]{inputenc}
\else % if luatex or xelatex
  \ifxetex
    \usepackage{mathspec}
  \else
    \usepackage{fontspec}
  \fi
  \defaultfontfeatures{Ligatures=TeX,Scale=MatchLowercase}
\fi
% use upquote if available, for straight quotes in verbatim environments
\IfFileExists{upquote.sty}{\usepackage{upquote}}{}
% use microtype if available
\IfFileExists{microtype.sty}{%
\usepackage{microtype}
\UseMicrotypeSet[protrusion]{basicmath} % disable protrusion for tt fonts
}{}
\newif\ifbibliography

% Prevent slide breaks in the middle of a paragraph:
\widowpenalties 1 10000
\raggedbottom

% \AtBeginPart{
%   \let\insertpartnumber\relax
%   \let\partname\relax
%   \frame{\partpage}
% }
% \AtBeginSection{
%   \ifbibliography
%   \else
%     \let\insertsectionnumber\relax
%     \let\sectionname\relax
%     \frame{\sectionpage}
%   \fi
% }
% \AtBeginSubsection{
%   \let\insertsubsectionnumber\relax
%   \let\subsectionname\relax
%   \frame{\subsectionpage}
% }

\setlength{\emergencystretch}{3em}  % prevent overfull lines
\providecommand{\tightlist}{%
  \setlength{\itemsep}{0pt}\setlength{\parskip}{0pt}}
\setcounter{secnumdepth}{0}

\usepackage{sympytex}
\usepackage{listings}

\title{Hidden Surface Removal}
\author{André Grüning, University of Surrey}
\date{\today}

\begin{document}
\frame{\titlepage}

\begin{frame}{Hidden Surface Removal}

{[}from Kleinberg, Tardos: ``Algorithm Design'' (our textbook){]}

\begin{block}{Background}

\emph{Hidden Surface Removal} is a problem in Computer graphics that
scarcely needs an introduction: when Woody is standing in front of Buzz,
you should be able to see Woody but not Buzz; when Buzz is standing in
front of Woody,\dots 

\end{block}

\end{frame}

\begin{frame}

\begin{block}{Elaboration}

The magic of hidden surface removal is that you can often compute things
faster than your intuition suggests. Here's a clean geometric example to
illustrate a basic speed-up that can be achieved. You are given \(n\)
nonvertical lines in the plane, labelled \(l_1, \dots l_n\) with the
\(i\) line specified by \(l_i(x) := m_ix + b_i\). We will make the
(simplifying) assumption that no three of the lines meet at a single
point. We will also assume that no two lines have the same slope, ie $m_i \neq m_j, \forall i\neq j$.

Which of the lines are visible by an observer high up in the \(y\)
direction (and hence need to be rendered)?

\begin{itemize}[<+->]
\tightlist
\item
  We say that line \(l_i\) is \emph{uppermost} at a given
  \(x\)-coordinate \(x_0\) if its \(y\)-coordinate is greater than
  \(y\)-coordinates of all other lines at \(x_0\).
\item
  We say line \(l_i\) is \emph{visible} if there is some
  \(x\)-coordinate at which it is uppermost -- intuitively some portion
  of it can be seen if you look down from \(y \to \infty\).
\end{itemize}

\end{block}

\end{frame}

\begin{frame}{Our programme}

\begin{block}{Our programme}

\begin{itemize}[<+->]
\tightlist
\item
  Understand the problem.
\item
  Try to find a brute force algorithm to gain a deeper understanding of the problem.
 \item
  Implementation: What data structure do we need to present a line?
\item
  Give an algorithm that takes \(n\) lines and in \(O(n\log n)\) time
  returns all of the ones that are visible.
\end{itemize}

\end{block}

\end{frame}

\section{Understand the problem}

\begin{frame}{Understand the problem}

\begin{itemize}
    \item We draw a couple of lines to understand what it means to be uppermost and to be visible. 
    \item (It's always good to try to find some visualisation to understand a new problem - and you should make that a habit.)
\end{itemize}
    
Before proceeding from this point we all should have a good understanding of what being uppermost and being visible means and why this is related to Hidden Surface Removal. If not, draw a few more different arrangements of straight lines which are partially or fully obscured by others when viewed from the top of your figures
\end{frame}

\section{Brute Force}

\begin{frame}{Brute Force Approach}

  Just take the problem literally:
  \begin{enumerate}
  \item To be visible, a line must be uppermost at least one point.
  \item Given a point $x_0$ a line is uppermost if it has the greatest $y$-coordinate.
  \item So we need to check at all points $x$ in a given range which line is uppermost.
  \end{enumerate}
\end{frame}

\begin{frame}{Brute Force Pseudocode}

  \begin{algorithm}[H]
    \caption{Removing Hidden Surfaces Brute Force}
    \begin{algorithmic}
      \Procedure{CalcVisibleSurfaces}{$L$: list of lines, $R \subset \mathbf{R} $: interval of interest for rendering}
      \State{$V = \emptyset$} \Comment{Set of visible lines}
      \For{$x \in R$}
      \State{$\text{uppermost} = \max\arg_{l \in L} l(y)$} \Comment{ $O(n)$ time}
      \State{Add uppermost to $V$}
      \EndFor
      \Return{$V$}
      \EndProcedure
    \end{algorithmic}
  \end{algorithm}
  \begin{itemize}
  \item Running time: Number of points in $R$ times O(n).
  \item But what is the size of $R$? If $R$ is an interval then there are an infinite number of points. $\Rightarrow$ Infinite running time\footnote{Of course if we only have finite resolution screen to render for, $R$ is not an interval in $\mathbf(R)$, but a set of discrete points -- eg the $x$-coordinates of 4000 pixels (assuming the screen is 4000 pixels wide. But still: what if we want to calculate this for an arbitrary yet unkown resultion, or just -- as we are in this module -- want to know the answers in abstractu.}
  \end{itemize}
\end{frame}

\begin{frame}{Frame Title}
    \begin{itemize}
        \item Clearly, this brute force algorithms is not feasible.
        \item Do we really need to calculate which line is uppermost at all $x \in R$?
        \item Or is there a smaller set of somehow "special" points?
        \begin{itemize}
            \item Q: Can what is the uppermost line change at any point, or only at special points? And what happens between these points?
        \end{itemize}
    \end{itemize}
\end{frame}

\section{Refined Brute Force}
\begin{frame}{Refined Brute Force}

\begin{itemize}[<+->]
    \item Observation: Uppermost line can only change at intersection points of any two lines.
    \item Therefore at any point \emph{between} two intersections the uppermost line will not change.
    \item Idea: Check only which is the uppermost line between any two intersections.\footnote{We could also check at the intersections points themselves, but that would need more case distinctions (why?)}
    \item What would be the running time of this refined brute-force algorithm?
        \begin{itemize}
            \item For $n$ lines there are $O(n^2)$ intersections\footnote{To be more precise it is $ (n-1) + (n-2) + (n-3) + \cdots + 2 + 1 = \sum_{i = 1}^{i < n} i = \frac{(n-1)(n-2)}{2}$ intersections.}, and therefore also $O(n^2)$ intervals between two neighbouring intersections.
            \item For each of these $O(n^2)$ intervals e need to pick a point to calculate the $y$ coordinate of $n$ lines.
            \item[$\Rightarrow$] $n \times O(n^2) = O(n^3)$ steps.
        \end{itemize}
        \end{itemize}
\end{frame}

\begin{frame}{What is missing?}
    What does this outline of the algorithm not specify?
    \begin{itemize}
        \item How to find the intervals.
        \item That the intersection points need to be sorted.
        \item Which point to choose in an interval.
        \item \dots
    \end{itemize}
\end{frame}

\begin{frame}{Refined Brute Force Pseudocode}

\begin{algorithm}[H]
\caption{Refined Brute Force}
\begin{algorithmic}
    \Procedure{CalcVisibleSurfaces}{L: list of lines}
    \State{$I$: list of $x$-coordinates of intersections}
        \For{any pair of indices $i,j$ with $i < j$} \Comment{ $O(n^2)$} 
        \State{Calculate the $x$-coordinate of intersection between $l_i$ and $l_j$}
        \State{Add this $x$ to $I$}
        \EndFor
        \State{$V = []$: list of visible lines}
        \State{Sort $I$ by increasing $x$} \Comment {$O(n^2\log n^2)$}
        \For{pairs of adjacent $x_1,x_2$ in $I$} \Comment{$O(n^2)$}
        \State{$x_0 = \frac{x_1 + x_2}{2}$}
        \State{$uppermost = \max\arg_{l \in L} l(x_0)$} \Comment{$O(n)$}
        \State{Add $uppermost$ to $V$}
        \EndFor
        \State{Remove duplicates from $V$} \Comment{$O(n \log n)$}
        \Return{V}
    \EndProcedure
\end{algorithmic}
\end{algorithm}
Note: The code does not depend on the size/length of the interval $R$ any longer.
\end{frame}

\begin{frame}{Problems with this code}

\begin{itemize}%[<+->]
    \item Is this code complete? 
      \begin{itemize}
      \item Actually a corner case is missing from the code. Which?
      \end{itemize}
    \item Test it on a small example!
    \item How do we remove duplicates?
    \item Why do we take a list and remove duplicates, and not a set?
\end{itemize}
\end{frame}

\begin{frame}{Implemention of the refined brute force algorithm.}
    We firstly need to think about the data structure to represent a
    line $l(x) = mx+b$ in code. Fields for a line object are from the
    above: 
    \begin{itemize}
    \item A line \emph{has} a slope $m$.
    \item A line \emph{has} an offset $b$
    \end{itemize}
    and we need a line object to do the following functions:
    \begin{itemize}
    \item return its value $y = l(x)$ for a given point $x$: method with signature \texttt{y(x)}
    \item return the $x$-coordinate of the intersection with another line: method with signature \texttt{intersection(Line other)}
    \end{itemize}
    The implementation of \texttt{y(x)} is straight forward. For
    method \texttt{intersection} we need to know how to calculate
    $x$-coordonate $x_0$ of the 
    intersetion of two lines $l_1(x) = m_1x + b_1$ and $l_2(x) = m_2 +
    b_2$. Basic algebra yields
    \begin{equation}
      x_0 = \frac{m_1 - m_2}{b_2 - b_1}
    \end{equation}
\end{frame}

\begin{frame}{Data structure representing a line}

\tiny
\lstinputlisting[language=Python, breaklines=true, firstline=4,
lastline=33 ]{HiddenSurfaceRemoval.py}
\end{frame}

\begin{frame}{Refined Brute Force -- Python implementation}
\tiny
\lstinputlisting[language=Python, breaklines=true, firstline=35,
lastline=62 ]{HiddenSurfaceRemoval.py}
\end{frame}

\begin{frame}{Running the algorithm}

We have now at least code that we can run on a set of straight lines
in order to get a better idea of what geometric properites might mark
the visibile lines.

\begin{block}{Code to generate 100 random lines and run the algorithm}
\tiny
\lstinputlisting[language=Python, breaklines=true, firstline=157,
lastline=167 ]{HiddenSurfaceRemoval.py}
\end{block}
\end{frame}



\begin{frame}[fragile]{100 straight lines}

  \begin{sympysilent}
    import HiddenSurfaceRemoval as hsr
    pL, pV = hsr.hundred_lines()
  \end{sympysilent}



  This is what 100 random straight lines look like:
  \begin{figure}
    \caption{100 random straight lines. Which are visible?}
    \sympyplot[width=\textwidth]{pL}
    \centering
  \end{figure}
\end{frame}

\begin{frame}{Visible Lines}

  \begin{figure}
    \caption{Visible lines among the random lines -- typically in the
      order of 10. (Note: not all sections of these lines are visible.}
    \sympyplot[width=\textwidth]{pV  }
    \centering
  \end{figure}

\end{frame}

\begin{frame}[fragile]{Visible Lines}

  \begin{sympysilent}
    pL.extend(pV)
  \end{sympysilent}

  \begin{figure}
    \caption{All lines with the visible ones in red.   On visual inspection, it appears the the lines with the steepest
  positive or negative slope tend to be visibile. Perhaps that a cue
  for a later fast algorithm?}
    \sympyplot[width=\textwidth]{pL}
    \centering
  \end{figure}
\end{frame}

\begin{frame}{Towards a fast algorithm}

\begin{itemize}[<+->]
    \item We got from the brute force algorithm to the refined
      algorithm by using geometric properties of the problem. 
    \item Can we find other properties of the problem that help reduce
      time complexity? 
    \item Do we really need to calculate the $y$-coordindates of
      \emph{all} lines at \emph{all} intersection points? 
    \item For both a Dynamic Programming approach or a Divide and
      Conquer approach\footnote{At this stage we don't know yet which
        of them could lead us to an algorithm for this problem.}, 
      what would be any base cases?
    \item Does it make sense to order the lines in a certain order?
      Which orders would be natural? 
\end{itemize}
\end{frame}

\begin{frame}{The simplest cases -- 0 and 1 lines}

In search of what could be base cases for a recurive algorithms
(either D\&C or Dynamic Progamming), it makes sense to look at the
simplest cases:

\begin{itemize}
\item No lines: then none are visibile
\item 1 line: then it's also visibile
\end{itemize}
So these cases are fairly trivial.
\end{frame}

\begin{frame}{The simplest cases -- 2 lines}

Two lines: As we assume that no two lines are parallel, they must
intersect somewhere, and are both visibile. So that's trivial as well.

\begin{tikzpicture}%[>=latex]
    \begin{axis}[
    %with=6cm,
    axis lines=middle,
    grid,
    grid style={densely dashed}, 
    xlabel=\(x\), ylabel=\(y\),
    every axis plot/.style={very thick},
    ]
    \addplot coordinates { (8,2) (-2,-3)};
    \addplot coordinates { (-2,-2) (8,1)};
    %\addplot coordinates { (-2,2) (8,-1)};
    \end{axis}
\end{tikzpicture}

But we can perhaps learn something useful for later: To the left of
the intersection point, the line with the smaller slope is uppermost
and to the right of it the one with the greater slope.
\end{frame}

\begin{frame}[fragile]{Three lines -- which are visible?}

  \begin{columns}
    \begin{column}{0.5\textwidth}
      \begin{tikzpicture}[scale=0.8]%[>=latex]
        \begin{axis}[
          % with=6cm,
          axis lines=middle,
          grid,
          grid style={densely dashed}, 
          xlabel=\(x\), ylabel=\(y\),
          every axis plot/.style={very thick},
          ]
          \addplot coordinates { (8,2) (-2,-3)};
          \addplot coordinates { (-2,-1.5) (8,1.5)};
          \addplot coordinates { (-2,2) (8,-1)};
        \end{axis}
      \end{tikzpicture}
      \caption{Three lines visible}
  \end{column}
  \pause
  \begin{column}{0.5\textwidth}
    
    \centering
    
    \begin{tikzpicture}[scale=0.8]%[>=latex]
      \begin{axis}[
        % with=6cm,
        axis lines=middle,
        grid,
        grid style={densely dashed}, 
    xlabel=\(x\), ylabel=\(y\),
    every axis plot/.style={very thick},
    ]
    \addplot coordinates { (8,2) (-2,-3)};
    \addplot coordinates { (-2,-2) (8,1)};
    \addplot coordinates { (-2,2) (8,-1)};
    \end{axis}
\end{tikzpicture}
\caption{Two lines visible}
\end{column}
\end{columns}
We observe (by drawing many more arrangements of three lines):
\scriptsize
\begin{itemize}[<-+>]
\item The lines with the greatest slope and the one with the smallest
  slopes are always visible. Let's call them $L_0$ and $L_2$.
\item The line $L_1$ with the middle slope is only visible if at the
  intersection $I_{02}$ of $L_0$ and $L_2$ it is above this
  intersection.\footnote{\tiny If $L_1$ is below $I_{02}$, then to the right of
    $I_{02}$, $L_2$ is above it and due to its greater slope will
    always be above. To the left of $I_{02}$, $L_0$ is above, and due
    to its smaller (more negative slope) will always stay above it.}
\end{itemize}

\end{frame}

\end{document}
